\documentclass[spec, och, pract, times]{shiza}
% параметр - тип обучения - одно из значений:
%    spec     - специальность
%    bachelor - бакалавриат (по умолчанию)
%    master   - магистратура
% параметр - форма обучения - одно из значений:
%    och   - очное (по умолчанию)
%    zaoch - заочное
% параметр - тип работы - одно из значений:
%    referat    - реферат
%    coursework - курсовая работа (по умолчанию)
%    diploma    - дипломная работа
%    pract      - отчет по практике
% параметр - включение шрифта
%    times    - включение шрифта Times New Roman (если установлен)
%               по умолчанию выключен

\usepackage{subfigure}
\usepackage{tikz,pgfplots}
\pgfplotsset{compat=1.5}
\usepackage{float}
\usepackage{minted}
\usepackage{ragged2e}
\justifying
%\usepackage{titlesec}
\setcounter{secnumdepth}{4}
%\titleformat{\paragraph}
%{\normalfont\normalsize}{\theparagraph}{1em}{}
%\titlespacing*{\paragraph}
%{35.5pt}{3.25ex plus 1ex minus .2ex}{1.5ex plus .2ex}

\titleformat{\paragraph}[block]
{\hspace{1.25cm}\normalfont}
{\theparagraph}{1ex}{}
\titlespacing{\paragraph}
{0cm}{2ex plus 1ex minus .2ex}{.4ex plus.2ex}
\renewcommand{\baselinestretch}{1.5}

% -----------------------------------------------------------------------------%


\usepackage[T2A]{fontenc}
\usepackage[utf8]{inputenc}
\usepackage{graphicx}
\graphicspath{ {./images/} }
\usepackage{tempora}

\usepackage[sort,compress]{cite}
\usepackage{amsmath}
\usepackage{amssymb}
\usepackage{amsthm}
\usepackage{fancyvrb}
\usepackage{listingsutf8}

\usepackage[english,russian]{babel}

%\usepackage[colorlinks=true]{hyperref}
\usepackage{url}
\newcommand{\eqdef}{\stackrel {\rm def}{=}}


\newtheorem{lem}{Лемма}

\begin{document}

% Кафедра (в родительном падеже)
\chair{теоретических основ компьютерной безопасности и криптографии}

% Тема работы
\title{отчет}

% Курс
\course{четвертый}

% Группа
\group{431}

% Факультет (в родительном падеже) (по умолчанию "факультета КНиИТ")
\department{факультета компьютерных наук и информационных технологий}

% Специальность/направление код - наименование
%\napravlenie{09.03.04 "--- Программная инженерия}
%\napravlenie{010500 "--- МОАИС}
%\napravlenie{230100 "--- Информатика и вычислительная техника}
%\napravlenie{231000 "--- Программная инженерия}
\napravlenie{090301 "--- Компьютерная безопасность}

% Для студентки. Для работы студента следующая команда не нужна.
% \studenttitle{Студентки}

% Фамилия, имя, отчество в родительном падеже
\author{Иванова Ксения Владиславовна}

% Заведующий кафедрой
\chtitle{д.~ф.-м.~н.,~доцент} % степень, звание
\chname{М.~Б.~Абросимов}

%Научный руководитель (для реферата преподаватель проверяющий работу)
\satitle{доцент} %должность, степень, звание
\saname{А.~В.~Гортинский}

% Руководитель практики от организации (только для практики,
% для остальных типов работ не используется)
% \patitle{к.ф.-м.н.}
% \paname{С.~В.~Миронов}

% Семестр (только для практики, для остальных
% типов работ не используется)
\term{восьмой}

% Наименование практики (только для практики, для остальных
% типов работ не используется)
% \practtype{Ознакомительная}

% Продолжительность практики (количество недель) (только для практики,
% для остальных типов работ не используется)
% \duration{2}

% Даты начала и окончания практики (только для практики, для остальных
% типов работ не используется)
% \practStart{22 июня 2020}
% \practFinish{05 июля 2020}

% Год выполнения отчета
\date{2024}

\maketitle

% Включение нумерации рисунков, формул и таблиц по разделам
% (по умолчанию - нумерация сквозная)
% (допускается оба вида нумерации)
% \secNumbering

%------------------------------------------------------------------------------%

\tableofcontents

\intro
В данной работе рассмотрим тему аппаратно"=программных модулей доверенной 
загрузки, актуальность, принцип работы, а также рассмотрим отечественные 
реализации модулей. Также проведем настройку домена для организации рекрутингового
агентства и проверим работу настройки системы управления доступом.

\section{Аппаратно"=программный модуль доверенной загрузки}
% Платформа для разработки "--- основа для приложения,-- пример тире
% его методы для определенных представлений~[10]. -- ссылка на источник
\subsection{Общее описание}
Введем несколько определений.

Процедура доверенной загрузки "--- это процесс загрузки системного программного
обеспечения после выполнения успешной аутентификации оператора изделия и 
исключительно с выбранного учтённого носителя, реализованный в доверенной среде [1].

Доверенная среда "--- это совокупность программно-технических средств и 
коммуникационных ресурсов, для которых однозначно определены состав, архитектура,
алгоритмы функционирования, правила обработки информации и в отношении которой 
верны следующие предположения:
\begin{itemize}
  \item[--] проведены исследования по требованиям нормативных документов по безопасности информации в объёме, согласованном с регулятором;
  \item[--] гарантирована её целостность и неизменность в составе изделия на период эксплуатации за счёт реализации соответствующих программно-технических и организационно-режимных мер.
\end{itemize}


Загрузка операционной системы происходит с жесткого диска, установленного в 
устройстве, такой способ открывает множество возможностей для различных 
хакерских атак и компрометации безопасности системы. Избежать потенциальной "встречи" 
со злоумышленниками можно, используя средства доверенной загрузки (СДЗ).

Средства доверенной загрузки представляют собой решения программно-аппаратные, 
обеспечивающие высокий уровень безопасности еще на самом раннем этапе загрузки 
операционной системы. Основная задача СДЗ "--- контроль целостности программного 
обеспечения и аппаратной конфигурации компьютера перед его запуском. Многие СДЗ 
также выполняют функции средств идентификации и аутентификации пользователей, что 
дополнительно повышает защищенность системы.

Комплексный подход к вопросам доверенной загрузки и применение соответствующих
программно-аппаратных решений позволяет значительно повысить уровень информационной
безопасности компьютерных систем еще до начала их основной эксплуатации, что является 
критически важным в условиях постоянно растущего количество компьютерных угроз.

Особенно актуальна реализация средств доверенной загрузки в критически важных 
системах, таких как банковские терминалы и другие корпоративные устройства. 
Злоумышленники часто пытаются внедрить вредоносное программное обеспечение, 
функционирующее еще до старта основной операционной системы, с целью дальнейшего 
проникновения в корпоративную сеть и компрометации безопасности. Использование 
СДЗ позволяет эффективно противостоять подобным атакам, гарантируя загрузку только
легитимного и проверенного ПО.

\newpage
\subsection{Принципы работы}

В настоящее время был проведен ряд научно-технических работ, по результатам которых
разработан новый подход, лишенный недостатков предыдущих решений. Для реализации 
данного подхода необходимо полное замещение проприетарного программного обеспечения 
BIOS с получением на него исходного кода. Это позволит встроить в BIOS функции 
модуля доверенной загрузки [2].

При этом ПО BIOS будет реализовывать исключительно базовые, минимально необходимые 
функции, достаточные для корректного функционирования процессорной платы с 
установленной операционной системой. Передача управления на загрузчик ОС будет 
осуществляться не BIOS, а самим ПМДЗ, что обеспечит доверенную и максимально 
быструю загрузку всей системы.

\begin{figure}[H]
  \centering
  \includegraphics[width=1\textwidth]{pict/8}
  \caption{Общий алгоритм работы ПО BIOS c функциями ПМДЗ}
  \label{fig:55}
\end{figure}

Данный подход позволяет устранить недостатки предыдущих решений и реализовать надежную и эффективную систему доверенной загрузки, интегрированную на низком уровне в базовое программное обеспечение вычислительной платформы.

Модули доверенной загрузки осуществляют верификацию контрольных сумм и других 
характеристик загружаемых данных, сравнивая их с некоторыми уже известными значениями. Таким образом обеспечивается надежный контроль целостности 
программного обеспечения еще до того, как оно будет фактически запущено. Это 
является ключевым фактором защиты от широкого спектра угроз, связанных с загрузкой 
вредоносных модификаций.

Применение средств доверенной загрузки, которые интегрированы в аппаратные платформы или
реализованы в виде отдельных программно-аппаратных устройств, позволяет повысить
уровень информационной безопасности критически важных систем, снизив риски успешных 
атак на этапе загрузки операционной системы.


\newpage
\subsection{Отечественные решения}
На российском рынке представлено множество решений этого класса, каждое из которых
обладает своим набором возможностей и характеристик. Рассмотрим несколько вариантов данных решений:

\subsubsection{«Аккорд-АМДЗ»}
Средство доверенной загрузки уровня платы расширения «Аккорд-АМДЗ» – разработка 
компании «ОКБ САПР» [3]. 

\begin{figure}[H]
  \centering
  \includegraphics[width=1\textwidth]{pict/17}
  \caption{Аппаратный модуль «Аккорд-АМДЗ»}
  \label{fig:60}
\end{figure}


«Аккорд-АМДЗ» "--- это аппаратный модуль доверенной загрузки
для IBM-совместимых ПК серверов и рабочих станций корпоративной сети. Продукт 
«ОКБ САПР» соответствует концепции резидентного компонента безопасности. Это 
автономное примитивное устройство с защищенной памятью, которое способно влиять 
на архитектуру старта устройства.

Данный защитный комплекс включает в себя специализированный
контроллер с предустановленной операционной средой. Помимо этого, 
СДЗ распространяется вместе с функциональным программным обеспечением. Оба этих
составляющих еще на этапе изготовления становятся частью единого ПО, 
размещенного в энергонезависимой флеш-памяти контроллера.

Комплекс создает доверенную среду для работы программ, обеспечивающих защиту 
на всех шагах (Рис. 3).

\begin{figure}[H]
  \centering
  \includegraphics[width=1\textwidth]{pict/9}
  \caption{Работа комплекса}
  \label{fig:56}
\end{figure}


Помимо этого, «Аккорд-АМДЗ» обеспечивает:
\begin{itemize}
  \item[--] защиту ресурсов СВТ от лиц, не допущенных к работе на ней;
  \item[--] аутентификацию пользователей с защитой от раскрытия пароля до загрузки ОС;
  \item[--] блокировку загрузки с внешних носителей;
  \item[--] контроль целостности технических, программных средств и объектов файловых систем, размещенных на динамических дисках;
  \item[--] доверенную загрузку системного и прикладного ПО;
  \item[--] регистрацию контролируемых событий в системном журнале, размещенном в энергонезависимой памяти контроллера;
  \item[--] возможность физической коммутации управляющих сигналов периферийных устройств;
  \item[--] администрирование встроенного ПО комплекса;
  \item[--] регистрацию, сбор, хранение и выдачу данных о событиях, происходящих в СВТ в части системы защиты от несанкционированного доступа.
\end{itemize}

У «Аккорд-АМДЗ» есть сертификаты образцов ФСТЭК. Также это СДЗ входит в реестр отечественного ПО.

\subsubsection{ПАК «Соболь»}
«Соболь» "--- аппаратно-программный модуль доверенной загрузки, функционирующий в среде UEFI. 
Электронный замок «Соболь» предназначен для защиты персональных компьютеров, 
специализированных устройств и серверов в их классическом понимании. Продукт компании 
«Код Безопасности» позволяет вести контроль неизменности файловых систем: NTFS, FAT16,
 FAT32, EXT2, EXT3, EXT4 в операционных системах Linux и Windows [4].

 \begin{figure}[H]
  \centering
  \includegraphics[width=1\textwidth]{pict/15}
  \caption{Аппаратный модуль ПАК «Соболь»}
  \label{fig:61}
\end{figure}


ПАК «Соболь» по заявлениям компании-разработчика создан для решения четырех ключевых задач:
\begin{itemize}
  \item [--] контроль целостности компонентов ИС;
  \item [--] запрет загрузки ОС с внешних носителей;
  \item [--] защита конфиденциальной информации и гостайны в соответствии с требованиями нормативных документов;
  \item [--] защита информации от несанкционированного доступа.
\end{itemize}

\begin{figure}[H]
  \centering
  \includegraphics[width=1\textwidth]{pict/11}
  \caption{Принцип работы модуля}
  \label{fig:57}
\end{figure}


Среди интересных преимуществ ПАК «Соболь» можно отметить наличие «сторожевого таймера». 
С его помощью СДЗ блокирует доступ к компьютеру, если управление при его включении 
не передано ПАК «Соболь». Вся информация о попытках входа в систему записывается 
устройством в журнал, который хранится в энергонезависимой памяти. Журнал фиксирует 
не только сам факт входа в систему, но и имя пользователя, предъявление незарегистрированных 
идентификаторов, неправильные вводы пароля, а также дату и время регистрации этих событий.
 Таким образом «Соболь» позволяет компании применить дополнительные меры защиты информации
 при обнаружении несанкционированных попыток входа в систему.

ПАК «Соболь» входит в реестр отечественного ПО. Также продукт имеет сертификаты соответствия
требованиям ФСТЭК и ФСБ.



\subsubsection{Dallas Lock}

Несмотря на американский мотив названия, Dallas Lock "--- разработка российской компании 
«Конфидент». Это средство доверенной загрузки представляет собой плату расширения 
для защиты информации. По заявлению «Конфидент», продукт может защищать сведения вплоть 
до уровня «совершенно секретно». Как и другие решения этого типа Dallas Lock ведет журнал 
и проверяет целостность программно-аппаратной среды до загрузки операционной системы [5].

\begin{figure}[H]
  \centering
  \includegraphics[width=1\textwidth]{pict/14}
  \caption{Аппаратные модули Dallas Lock}
  \label{fig:59}
\end{figure}

Dallas Lock обладает несколькими преимуществами:
\begin{itemize}
  \item[--] администрировать средство доверенной загрузки можно без использования ресурсов загружаемой штатной ОC;
  \item[--] позволяет разграничивать доступ к управлению СДЗ;
  \item[--] СДЗ поддерживает безопасный режим загрузки UEFI;
  \item[--] имеет собственные часы с независимым источником питания;
  \item[--] хранит ключи и служебную информацию в энергонезависимой памяти платы СДЗ;
  \item[--] доступ к СДЗ ограничен после загрузки штатной операционной системы;
\end{itemize}

Решение позволяет использовать двухфакторную аппаратную аутентификацию с популярными USB-ключами и электронными ключами.
\begin{figure}[H]
  \centering
  \includegraphics[width=1\textwidth]{pict/12}
  \caption{Комплекс защиты Dallas Lock}
  \label{fig:58}
\end{figure}

СДЗ Dallas Lock входит в реестр отечественного ПО, а также обладает сертификатами ФСТЭК и Минобороны России.

\subsubsection{ViPNet SafeBoot}

Данный продукт разработан российской компанией «ИнфоТеКС». Высокотехнологичный 
программный модуль доверенной загрузки устанавливается в UEFI BIOS. Этот продукт классифицируется как 
СДЗ уровня базовой системы ввода-вывода. ViPNet SafeBoot предназначен для защиты персональных компьютеров, 
мобильных устройств и серверов. Это решение, как и другие, защищает от угрозы несанкционированного 
доступа на этапе загрузки операционной системы [6].

\begin{figure}[H]
  \centering
  \includegraphics[width=1\textwidth]{pict/13}
  \caption{Аппаратный модуль ViPNet SafeBoot}
  \label{fig:62}
\end{figure}

VipNet SafeBoot обладает несколькими возможностями, среди которых:
\begin{itemize}
  \item [--] авторизация в AD/LDAP;
  \item [--] поддержка SSO для входа в ОС;
  \item [--] защита от вредоносного ПО в BIOS;
  \item [--] ведение журнала событий безопасности;
  \item [--] наличие шаблонов администрирования, позволяющих быстро настроить СДЗ;
  \item [--] защита от обхода и самотестирование;
  \item [--] запрет загрузки с нештатных и внешних носителей;
  \item [--] поддержка двухфакторной аутентификации.
\end{itemize}

Также компания «ИнфоТеКС» среди преимуществ собственного продукта отмечает неизвлекаемость ПМДЗ, 
в отличие от других решений этого класса. VipNet SafeBoot входит в реестр отечественного ПО,
также продукт сертифицирован ФСТЭК России, а значит удовлетворяет требования регуляторов, 
также может быть использовать для построения ИСПДн до У31, ГИС, АСУ ТП и КИИ до 1 класса 
защищенности.




\section{Настройка подсистемы}
\subsection{Вариант}
Вариант №9. Подсистема управления доступом для системы рекрутинговой организации, которая 
обрабатывает ПДн работников и специальные данные 100001 клиентов. 

Определим следующие пункты для составления итогового списка требований:
\begin{itemize}
  \item[1.] класс АC, базовые требования;
  \item[2.] свойства информации;
  \item[3.] класс защищенности СВТ;
  \item[4.] ГИС или ИСПДн; 
  \item[5.] уровень защищенности ИС.
\end{itemize}

%  Определение 
%  -- класс АC, базовые требования (1Г)
%  -- Класс защищенности СВТ (5)
%  -- Определить ГИС или ИСПДн и наверное требования к нему
%  -- Уровень защищенности ИС (2)
 

\subsection{Теоретическая часть}
\subsubsection{Класс АС и базовые требования к подсистеме}
Согласно документу <<Автоматизированные системы. Защита от несанкционированного
доступа к информации. Классификация автоматизированных систем
и требования по защите информации>>, пункт 1.9, система  имеет класс \textbf{1Г} т.к она 
обрабатывает персональные данные и специальные, также является многопользовательской системой с разными полномочиями [7].

В пункт 2.10 документа выделяются следующее базовые требование к подсистемам для данного класса (Рис. 9).

\begin{figure}[H]
  \centering
  \includegraphics[width=1.1\textwidth]{pict/20}
  \caption{Требования к подсистемам}
  \label{fig:1}
\end{figure}

\textbf{Подсистема управления доступом:}
\begin{itemize}
  \item[--] должна осуществляться идентификация и проверка подлинности субъектов доступа при входе в систему по идентификатору 
  (коду) и паролю условно-постоянного действия, длиной не менее шести буквенно-цифровых символов;
  \item[--] должна осуществляться идентификация терминалов, ЭВМ, узлов сети ЭВМ, каналов связи, внешних устройств ЭВМ по логическим именам;
  \item[--] должна осуществляться идентификация программ, томов, каталогов, файлов, записей, полей записей по именам;
  \item[--] должен осуществляться контроль доступа субъектов к защищаемым ресурсам в соответствии с матрицей доступа.
\end{itemize}
\textbf{Подсистема регистрации и учета:}

Должна осуществляться регистрация входа (выхода) субъектов доступа в
систему (из системы), либо регистрация загрузки и инициализации
операционной системы и ее программного останова. Регистрация выхода из
системы или останова не проводится в моменты аппаратурного отключения
АС. В параметрах регистрации указываются:
\begin{itemize}
  \item[--] дата и время входа (выхода) субъекта доступа в систему (из системы) или
  загрузки (останова) системы;
  \item[--] результат попытки входа: успешная или неуспешная - несанкционированная;
  \item[--] идентификатор (код или фамилия) субъекта, предъявленный при попытке
  доступа;
  \item[--] код или пароль, предъявленный при неуспешной попытке;
  \item[--] должна осуществляться регистрация выдачи печатных (графических) 
  документов на "твердую" копию. В параметрах регистрации указываются:
  дата и время выдачи (обращения к подсистеме вывода);
  спецификация устройства выдачи [логическое имя (номер) внешнего устройства]; 
  краткое содержание (наименование, вид, шифр, код) и уровень конфиденциальности
  документа;
  идентификатор субъекта доступа, запросившего документ;
  \item[--] должна осуществляться регистрация запуска (завершения) программ и процессов
  (заданий, задач), предназначенных для обработки защищаемых файлов. В параметрах 
  регистрации указываются: дата и время запуска, имя (идентификатор) программы (процесса, задания),
  идентификатор субъекта доступа, запросившего программу (процесс, задание), результат запуска (успешный, неуспешный - несанкционированный);
  \item[--] должна осуществляться регистрация попыток доступа средств 
  к защищаемым файлам. В параметрах регистрации указываются: дата и время попытки 
  доступа к защищаемому файлу с указанием ее результата: успешная, неуспешная - несанкционированная,
  идентификатор субъекта доступа, спецификация защищаемого файла;
  \item[--] должна осуществляться регистрация попыток доступа средств к следующим 
  дополнительным защищаемым объектам доступа: терминалам, ЭВМ, узлам сети ЭВМ, линиям (каналам) связи, 
  внешним устройствам ЭВМ, программам, томам, каталогам, файлам, записям, полям записей. В параметрах регистрации указываются:
  дата и время попытки доступа к защищаемому объекту с указанием ее результата: успешная, неуспешная - несанкционированная;
  идентификатор субъекта доступа;
  спецификация защищаемого объекта [логическое имя (номер)];
  \item[--] должен проводиться учет всех защищаемых носителей информации с помощью их
  маркировки и с занесением учетных данных в журнал (учетную карточку);
  \item[--] учет защищаемых носителей должен проводиться в журнале (картотеке) с регистрацией
   их выдачи (приема);
  \item[--] должна осуществляться очистка (обнуление, обезличивание) освобождаемых 
  областей оперативной памяти ЭВМ и внешних накопителей. Очистка осуществляется однократной 
  произвольной записью в освобождаемую область памяти, ранее использованную для хранения 
  защищаемых данных (файлов);
\end{itemize}

\textbf{Подсистема обеспечения целостности:}

Должна быть обеспечена целостность программных средств СЗИ НСД,
обрабатываемой информации, а также неизменность программной среды. При этом:

\begin{itemize}
  \item[--] целостность СЗИ НСД проверяется при загрузке системы по контрольным
  суммам компонент СЗИ;
  \item[--] целостность программной среды обеспечивается использованием
  трансляторов с языков высокого уровня и отсутствием средств модификации
  объектного кода программ в процессе обработки и (или) хранения
  защищаемой информации;
  \item[--] должна осуществляться физическая охрана СВТ (устройств и носителей
  информации), предусматривающая контроль доступа в помещения АС
  посторонних лиц, наличие надежных препятствий для несанкционированного
  проникновения в помещения АС и хранилище носителей информации,
  особенно в нерабочее время;
  \item[--] должно проводиться периодическое тестирование функций СЗИ НСД при
  изменении программной среды и персонала АС с помощью тест-программ,
  имитирующих попытки НСД;
  \item[--] должны быть в наличии средства восстановления СЗИ НСД,
  предусматривающие ведение двух копий программных средств СЗИ НСД и
  их периодическое обновление и контроль работоспособности.
\end{itemize}

\subsubsection{Свойства информции}

Так как система обрабатывает специальные персональные данные пользователей, а также 
данные своих сотрудников, что по Указе Президента РФ от 6 марта 1997 г. № 188 
"Об утверждении перечня сведений конфиденциального характера", персональные данные являются конфиденциальной информацией 
и должны защищаться, поэтому вместе с целостностью и доступностью необходимо обеспечить конфиденциальность [8].

\subsubsection{Класс защищенности СВТ}

Согласно документу <<РУКОВОДЯЩИЙ ДОКУМЕНТ. Средства вычислительной техники.
Защита от несанкционированного доступа к информации. Показатели
защищенности от несанкционированного доступа к информации>>, пункт 2.1.1 можно 
определить \textbf{5} класс защищенности СВТ [9].
Поскольку в отличие от 6 класса, 5 требует наличия контроля целостности средств 
СЗИ, что соответствует требованиям 
из 1.2.1, но в отличие от 4 класса, 5 не требует защиту работы с физическим носителем,
 сопоставления пользователя с устройством и изоляцию модулей (Рис. 10).

\begin{figure}[H]
  \centering
  \includegraphics[width=0.7\textwidth]{pict/3}
  \caption{Синий -- требования к системе, зеленый -- к подсистеме управления доступом и идентификации}
  \label{fig:3}
\end{figure}

\textbf{Дискреционный принцип контроля доступа}

КСЗ должен контролировать доступ наименованных субъектов
(пользователей) к наименованным объектам (файлам, программам, томам и
т.д.). Для каждой пары (субъект – объект) в СВТ должно быть задано явное и
недвусмысленное перечисление допустимых типов доступа (читать, писать и
т.д.), т.е. тех типов доступа, которые являются санкционированными для
данного субъекта (индивида или группы индивидов) к данному ресурсу СВТ
(объекту).

КСЗ должен содержать механизм, претворяющий в жизнь дискреционные
правила разграничения доступа.

Контроль доступа должен быть применим к каждому объекту и каждому
субъекту (индивиду или группе равноправных индивидов).

Механизм, реализующий дискреционный принцип контроля доступа,
должен предусматривать возможности санкционированного изменения ПРД,
в том числе возможность санкционированного изменения списка
пользователей СВТ и списка защищаемых объектов.

Права изменять ПРД должны предоставляться выделенным субъектам
(администрации, службе безопасности и т.д.).

Дополнительно должны быть предусмотрены средства управления,
ограничивающие распространение прав на доступ.

\textbf{Очистка памяти}

При первоначальном назначении или при перераспределении внешней
памяти КСЗ должен предотвращать доступ субъекту к остаточной
информации.

\textbf{Идентификация и аутентификация}

КСЗ должен требовать от пользователей идентифицировать себя при
запросах на доступ. КСЗ должен подвергать проверке подлинность
идентификации – осуществлять аутентификацию. КСЗ должен располагать
необходимыми данными для идентификации и аутентификации. КСЗ должен
препятствовать доступу к защищаемым ресурсам неидентифицированных
пользователей и пользователей, подлинность идентификации которых при
аутентификации не подтвердилась.

\textbf{Гарантии проектирования}

На начальном этапе проектирования СВТ должна быть построена модель
защиты. Модель должна включать в себя ПРД к объектам и
непротиворечивые правила изменения ПРД.

\textbf{Регистрация}

КСЗ должен быть в состоянии осуществлять регистрацию следующих
событий:
\begin{itemize}
  \item [--] использование идентификационного и аутентификационного механизма;
  \item [--] запрос на доступ к защищаемому ресурсу (открытие файла, запуск
  программы и т.д.);
  \item [--] создание и уничтожение объекта;
  \item [--] действия по изменению ПРД.
\end{itemize}

Для каждого из этих событий должна регистрироваться следующая
информация:
\begin{itemize}
  \item [--] дата и время;
  \item [--] субъект, осуществляющий регистрируемое действие;
  \item [--] тип события (если регистрируется запрос на доступ, то следует отмечать
   объект и тип доступа);
  \item [--] успешно ли осуществилось событие (обслужен запрос на доступ или нет).
\end{itemize}
КСЗ должен содержать средства выборочного ознакомления с
регистрационной информацией.

\textbf{Целостность КСЗ}

В СВТ пятого класса защищенности должны быть предусмотрены
средства периодического контроля за целостностью программной и
информационной части КСЗ.

\textbf{Тестирование}

В СВТ пятого класса защищенности должны тестироваться:
\begin{itemize}
  \item [--] реализация ПРД (перехват явных и скрытых запросов на доступ,
  правильное распознавание санкционированных и несанкционированных
  запросов, средства защиты механизма разграничения доступа,
  санкционированные изменения ПРД);
  \item [--] успешное осуществление идентификации и аутентификации, а также их
  средства защиты;
  \item [--] очистка памяти в соответствии с п. 2.3.2;
  \item [--] регистрация событий в соответствии с п. 2.3.5, средства защиты
  регистрационной информации и возможность санкционированного 
  ознакомления с ней;
  \item [--] работа механизма, осуществляющего контроль за целостностью КСЗ.
\end{itemize}

\textbf{Руководство пользователя}

Документация на СВТ должна включать в себя краткое руководство для
пользователя с описанием способов использования КСЗ и его интерфейса с
пользователем..

\textbf{Руководство по КСЗ}

Данный документ адресован администратору защиты и должен
содержать:

\begin{itemize}
  \item [--] описание контролируемых функций;
  \item [--] руководство по генерации КСЗ;
  \item [--] описания старта СВТ, процедур проверки правильности старта, процедур
  работы со средствами регистрации.
\end{itemize}


\textbf{Тестовая документация}

Должно быть предоставлено описание тестов и испытаний, которым
подвергалось СВТ (в соответствии с требованиями п.2.3.7), и результатов
тестирования.

\textbf{Конструкторская и проектная документация}

Должна содержать:
\begin{itemize}
  \item [--] описание принципов работы СВТ;
  \item [--] общую схему КСЗ;
  \item [--] описание интерфейсов КСЗ с пользователем и интерфейсов модулей КСЗ;
  \item [--] модель защиты;
  \item [--] описание механизмов контроля целостности КСЗ, очистки памяти,
  идентификации и аутентификации.
\end{itemize}


\subsubsection{ГИС или ИСПДн}

Так как моя организация "---рекрутинговое агентство, система которого обрабатывает специальные ПДн и сотрудников,
что не является государственной информацией, то система является ИСПДн.


\subsubsection{Уровень защищенности ИС}
Согласно постановлению от 1 ноября 2012 г. N 1119 <<ОБ УТВЕРЖДЕНИИ ТРЕБОВАНИЙ
К ЗАЩИТЕ ПЕРСОНАЛЬНЫХ ДАННЫХ ПРИ ИХ ОБРАБОТКЕ В ИНФОРМАЦИОННЫХ СИСТЕМАХ ПЕРСОНАЛЬНЫХ ДАННЫХ>>,
пункт 10, уровень защищенности можно определить как \textbf{УЗ 2} [10]. Так как система
подходит под одно из условий этого уровня, системе не грозят угрозы 1 и 2 класса, т.к используется лицензированное
ПО и ОС (Рис. 11).

\begin{figure}[H]
  \centering
  \includegraphics[width=1.1\textwidth]{pict/6}
  \caption{Условия для УЗ 2}
  \label{fig:4}
\end{figure}

Требования, которы выставляются для данного уровня защищенности, пункты 13-15 (Рис. 12). 

\begin{figure}[H]
  \centering
  \includegraphics[width=1\textwidth]{pict/5}
  \caption{Требования для УЗ 2}
  \label{fig:5}
\end{figure}

\subsubsection{Требования для подсистемы управления доступом}
По итогу собранных требований, можно выделить список требований конкретно для
подсистемы управления доступом:

\begin{itemize}
  \item[1.] идентификация и проверка подлинности субъектов доступа при входе в систему по идентификатору 
  (коду) и паролю условно-постоянного действия, длиной не менее шести буквенно-цифровых символов;
  \item[2.] идентификация терминалов, ЭВМ, узлов сети ЭВМ, каналов связи, внешних устройств ЭВМ по логическим именам;
  \item[3.] идентификация программ, томов, каталогов, файлов, записей, полей записей по именам;
  \item[4.] контроль доступа субъектов к защищаемым ресурсам в соответствии с матрицей доступа;
  \item[5.] КСЗ должен содержать механизм, претворяющий в жизнь дискреционные
  правила разграничения доступа;
  \item[6.] ограничение прав для пользователей, могут пенять права доступа только админы;
  \item[7.] настройка доступа к электронному журналу сообщений только для субъектов, использующих их в работе.
\end{itemize}
\newpage



1) Управление 
учетными записями пользователей, в том числе внешних пользователей;
5) Ограничение неуспешных попыток входа в информационную
систему (доступа к информационной системе);
6) Блокирование сеанса доступа в информационную систему после
установленного времени бездействия (неактивности) пользователя или по его
запросу;



% Вручную оформление ссылок не допускается

% \begin{figure}[H]
%   \centering
%   \includegraphics[width=1\textwidth]{pict/2}
%   \caption{принцип взаимодействия уровней приложения}
%   \label{fig:2}
% \end{figure}



\subsection{Практическая часть}
\subsubsection{Настройка доменной системы}
В практике используются виртуальные машины Windows Server 2003 и Windows 10, 
заранее присоединим компьютер к домену, и далее создадим подразделения для
нашей организации соответственно варианту.

Для нашей структуры создадим матрицу доступа субъектов к объектам.
\begin{figure}[H]
  \centering
  \includegraphics[width=1\textwidth]{pict/102}
  \caption{Матрица доступа, r - чтение, w - запись, c - изменение}
  \label{fig:101}
\end{figure}

\begin{figure}[H]
  \centering
  \includegraphics[width=1\textwidth]{pict/prac/54}
  \caption{}
\end{figure}

В домене создадим подразделения соответствующие варианту, а также на одном уровне с ними создадим 
главного пользователя.

\begin{figure}[H]
  \centering
  \includegraphics[width=1\textwidth]{pict/prac/3}
  \caption{Подразделения}
  \label{fig:11}
\end{figure}

Одним из требований к классу АС является установка для пользователя пароля длинной не менее 6 
цифро-буквенных символов, включим эту настройку.
\begin{figure}[H]
  \centering
  \includegraphics[width=1\textwidth]{pict/7}
  \caption{Парольная политика}
  \label{fig:19}
\end{figure}

Создадим пользователей в каждом подразделении.
\begin{figure}[H]
  \centering
  \includegraphics[width=1\textwidth]{pict/prac/1}
  \caption{Создание пользователя}
  \label{fig:12}
\end{figure}

\begin{figure}[H]
  \centering
  \includegraphics[width=1\textwidth]{pict/prac/2}
  \caption{Задание пароля пользователю}
  \label{fig:14}
\end{figure}

Проверим работу настройки сложности пароля, поменяем пользователю пароль.
\begin{figure}[H]
  \centering
  \includegraphics[width=1\textwidth]{pict/prac/50}
  \caption{Поменяем пароль на более короткий}
  \label{fig:50}
\end{figure}

\begin{figure}[H]
  \centering
  \includegraphics[width=0.9\textwidth]{pict/prac/51}
  \caption{Реакция системы на смену пароля}
  \label{fig:51}
\end{figure}

Создадим настройку ограниченного доступа к журналу, соответствующую требованию к уровню защищенности ИС.
\begin{figure}[H]
  \centering
  \includegraphics[width=1\textwidth]{pict/prac/59}
  \caption{Ограничение доступа к журналу}
\end{figure}

Зайдем в аккаунт одного оз пользователей и попробуем посмотреть журнал безопасность.
\begin{figure}[H]
  \centering
  \includegraphics[width=1\textwidth]{pict/prac/80}
  \caption{Реакция на доступ к журналу}
\end{figure}

Также чтобы в систему имели доступ только аутентифицированные пользователи, отключим учетную запись гостя.

\begin{figure}[H]
  \centering
  \includegraphics[width=1\textwidth]{pict/prac/60}
  \caption{Запрет гостевой учетной записи}
\end{figure}

Проверим работ предыдущей политики.
\begin{figure}[H]
  \centering
  \includegraphics[width=1\textwidth]{pict/prac/65}
  \caption{Реакция системы на вход гостя}
\end{figure}

\begin{figure}[H]
  \centering
  \includegraphics[width=1\textwidth]{pict/prac/66}
  \caption{Реакция системы на вход гостя}
\end{figure}

Аналогично создадим всех остальных пользователей для каждого подразделения.

\begin{figure}[H]
  \centering
  \includegraphics[width=0.9\textwidth]{pict/prac/4}
  \caption{Аккаунты работников}
  \label{fig:15}
\end{figure}

Создадим сетевой ресурс для общей папки во всех отделах, так как папка нужна для обмена данными, то 
все имеют право туда писать, читать, но изменять содержимое чужих файлов нет.

\begin{figure}[H]
  \centering
  \includegraphics[width=0.9\textwidth]{pict/prac/5}
  \caption{Общая папка}
  \label{fig:16}
\end{figure}

\begin{figure}[H]
  \centering
  \includegraphics[width=0.9\textwidth]{pict/prac/6}
  \caption{Права на общую папку}
  \label{fig:17}
\end{figure}


Также создадим в каждом подразделении будет сетевой ресурс, указывающий на общую папку.
\begin{figure}[H]
  \centering
  \includegraphics[width=0.9\textwidth]{pict/prac/7}
  \caption{Общие папки}
  \label{fig:18}
\end{figure}

Для своей папки, сотрудник имеет следующие права:
\begin{figure}[H]
  \centering
  \includegraphics[width=0.9\textwidth]{pict/prac/8}
  \caption{Разрешено}
  \label{fig:20}
\end{figure}


\begin{figure}[H]
  \centering
  \includegraphics[width=0.9\textwidth]{pict/prac/61}
  \caption{Запрещено}
\end{figure}


Как мы видим полный доступ к папке имеет только администратор, он может менять владельца, права на доступ и т.д,
 что отвечает требованию к классу защищенности СВТ по ограничению прав для пользователей.
\begin{figure}[H]
    \centering
    \includegraphics[width=1\textwidth]{pict/prac/9}
    \caption{Права на папку сотрудника}
    \label{fig:21}
\end{figure}

\begin{figure}[H]
  \centering
  \includegraphics[width=1\textwidth]{pict/prac/10}
  \caption{Права на папку начальника}
  \label{fig:22}
\end{figure}

\newpage
К папке главного менеджера доступ и права имеет только сам менеджер.
\begin{figure}[H]
  \centering
  \includegraphics[width=1\textwidth]{pict/prac/11}
  \caption{Права на папку главного менеджера}
  \label{fig:48}
\end{figure}
Настроем аналогичные правила для аккаунтов всех остальных подразделений.

\newpage


\subsubsection{Проверка работы настроек}

Зайдем в учетную запись работника кадров, проверим доступность.
\begin{figure}[H]
  \centering
  \includegraphics[width=0.9\textwidth]{pict/prac/25}
  \caption{Аккаунт работника кадров}
  \label{fig:24}
\end{figure}

\begin{figure}[H]
  \centering
  \includegraphics[width=0.9\textwidth]{pict/prac/26}
  \caption{Работник кадров -> Работница кадров}
  \label{fig:25}
\end{figure}


\begin{figure}[H]
  \centering
  \includegraphics[width=1\textwidth]{pict/prac/32}
  \caption{Работник кадров -> Работник юристов}
  \label{fig:31}
\end{figure}


\begin{figure}[H]
  \centering
  \includegraphics[width=1\textwidth]{pict/prac/27}
  \caption{Работник кадров -> Начальник кадров}
  \label{fig:26}
\end{figure}


\begin{figure}[H]
  \centering
  \includegraphics[width=1\textwidth]{pict/prac/31}
  \caption{Работник кадров -> Главный менеджер}
  \label{fig:30}
\end{figure}

\begin{figure}[H]
  \centering
  \includegraphics[width=1\textwidth]{pict/prac/28}
  \caption{Работник кадров -> Работник кадров}
  \label{fig:27}
\end{figure}


\begin{figure}[H]
  \centering
  \includegraphics[width=1\textwidth]{pict/prac/29}
  \caption{Может создавать документы}
  \label{fig:28}
\end{figure}

\begin{figure}[H]
  \centering
  \includegraphics[width=1\textwidth]{pict/prac/30}
  \caption{Созданный работником документ на сервере}
  \label{fig:29}
\end{figure}


\begin{figure}[H]
  \centering
  \includegraphics[width=1\textwidth]{pict/prac/33}
  \caption{Создание документа в общей папке}
  \label{fig:32}
\end{figure}

\begin{figure}[H]
  \centering
  \includegraphics[width=1\textwidth]{pict/prac/34}
  \caption{Создание документа в общей папке}
  \label{fig:33}
\end{figure}

Рассмотрим аккаунт начальника подразделения соискателей и его доступ.
\begin{figure}[H]
  \centering
  \includegraphics[width=1\textwidth]{pict/prac/35}
  \caption{Начальник соискателей -> Работник соискателей}
  \label{fig:34}
\end{figure}

\begin{figure}[H]
  \centering
  \includegraphics[width=1\textwidth]{pict/prac/36}
  \caption{Изменение документа в папке своего работника}
  \label{fig:35}
\end{figure}

\begin{figure}[H]
  \centering
  \includegraphics[width=1\textwidth]{pict/prac/39}
  \caption{Начальник соискателей -> Работник юрист}
  \label{fig:38}
\end{figure}

\begin{figure}[H]
  \centering
  \includegraphics[width=1\textwidth]{pict/prac/37}
  \caption{Начальник соискателей -> Начальник соискателей}
  \label{fig:36}
\end{figure}

\begin{figure}[H]
  \centering
  \includegraphics[width=1\textwidth]{pict/prac/38}
  \caption{Начальник соискателей -> Начальник кадров}
  \label{fig:37}
\end{figure}


\begin{figure}[H]
  \centering
  \includegraphics[width=1\textwidth]{pict/prac/40}
  \caption{Начальник соискателей -> Главный менеджер}
  \label{fig:39}
\end{figure}


\begin{figure}[H]
  \centering
  \includegraphics[width=1\textwidth]{pict/prac/42}
  \caption{Доступ к общей папке}
  \label{fig:41}
\end{figure}

\begin{figure}[H]
  \centering
  \includegraphics[width=1\textwidth]{pict/prac/44}
  \caption{Отказ в изменении файла}
  \label{fig:43}
\end{figure}

Аккаунт главного менеджера, который может заглядывать в любые папки.
\begin{figure}[H]
  \centering
  \includegraphics[width=1\textwidth]{pict/prac/45}
  \caption{Главный менеджер -> Начальник соискателей}
  \label{fig:44}
\end{figure}

\begin{figure}[H]
  \centering
  \includegraphics[width=1\textwidth]{pict/prac/46}
  \caption{Изменять папку он также не может}
  \label{fig:45}
\end{figure}

\begin{figure}[H]
  \centering
  \includegraphics[width=1\textwidth]{pict/prac/47}
  \caption{Главный менеджер -> Главный менеджер}
  \label{fig:46}
\end{figure}




% \addcontentsline{toc}{section}{\hspace{15mm}ЗАКЛЮЧЕНИЕ}
\section*{ЗАКЛЮЧЕНИЕ}
\addcontentsline{toc}{section}{ЗАКЛЮЧЕНИЕ}
Рассмотренные отечественные решения АПМДЗ, обладают широким комплексом 
полезных функций, который обеспечивает защиту не только на уровне загрузки ОС, 
но также и после. Данные модули к тому же имеют сертификат ФСТЭК, 
а некоторые внесены в реестр отечественного ПО, что позволят поднять уровень информационной 
безопасности в компании. 

Также была реализована настройка учетных записей в домене организации рекрутингового агентства,
и соответственно реализованы требования к данной системе, выполнение которых также было проверено.


\section*{СПИСОК ИСПОЛЬЗОВАННЫХ ИСТОЧНИКОВ}
\addcontentsline{toc}{section}{СПИСОК ИСПОЛЬЗОВАННЫХ ИСТОЧНИКОВ}
% \input{sections/istoch.tex}
% \section{Приложение А}
% \section{Приложение Б}
% \begin{thebibliography}{99}
  \bibitem{Andriyanov} Андриянов В., ОБЕСПЕЧЕНИЕ ИНФОРМАЦИОННОЙ БЕЗОПАСНОСТИ БИЗНЕСА, M.: Паблишер, 2010. -235 с.
  \bibitem{Gruzdev} Груздев С., Сабанов А., ``Вопросы идентификации и аутентификации при разработке мультиаппликационных карт'', [Электронный ресурс] -  URL:\\ https://www.aladdin-rd.ru/company/pressroom/articles/voprosy\_identifikacii\_\\i\_autentifikacii\_pri\_razrabotke\_multiapplikacionnyh\_kart, дата обращения 26.05.2024, Яз. Рус.
  \bibitem{Korzhov} Коржов В., ``Пароль на минуту'', [Электронный ресурс] / URL: \\https://www.osp.ru/cw/2005/01/84855/, дата обращения 26.05.2024, Яз. Рус.
  \bibitem{Sabanov} Сабанов А., ``О технологиях идентификации и аутентификации'', [Электронный ресурс] / URL: https://www.aladdin-rd.ru/company/pressroom/articles/o\_\\tehnologiah\_identifikacii\_i\_autentifikacii, дата обращения  26.05.2024, Яз. Рус.
  \bibitem{Fisenko} Фисенко Л., ``Новое лицо киберпреступности'', [Электронный ресурс] / URL:\\https://www.itweek.ru/infrastructure/article/detail.php?ID=71779, дата обращения 24.05.2024, Яз. Рус.
  \bibitem{Fiso} PHP Framework List: An Ultimate Guide to 102 PHP Frameworks for Web Developers, [Электронный ресурс] / URL: https://www.temok.com/blog/php-framework-list/, дата обращения 26.05.2024.
  \bibitem{Белик} Венц К., ``Безопасность ASP.NET Core'', пер. с англ. Д. А. Беликова, г. Москва, Издательство ДМК Пресс, 2023 г., Яз. Рус.
  \bibitem{Юртанова} Юртанова Е., Разработка web-приложений с использованием языка PHP / Е. М. Юртанова // Учебный эксперимент в образовании. 2011. 47 с.
  \bibitem{Laravel} ``Почему стоит выбрать Laravel для веб-разработки в 2022'', [Электронный ресурс] / URL: https://firecode.ru/blog/pochemu-stoit-vybrat-laravel-dlya-veb-razrabotki-v-2022, дата обращения 25.05.2024.
  \bibitem{Захаров} Захаров В., ``Проблемы выбора языков программирования при разработке кроссплатформенных приложений'', [Электронный ресурс] /  URL: https://cyberleninka.ru/article/n/problemy-vybora-yazykov-programmirovaniya-pri-razrabotke-krossplatformennyh-prilozheniy, дата обращения 25.05.2024.
\end{thebibliography}
% Неправильно оформлена библиография, нужен бибтех

\end{document}